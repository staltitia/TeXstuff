\documentclass[letterpaper]{article}\usepackage[margin=0.5in]{geometry}
%\usepackage{graphicx}
\begin{document}

\normalsize
\flushleft

\begin{minipage}[t]{0.48\textwidth}
\huge{\textbf{Daniel Hon Kit Chong}} \\ %center and resize this in future
\Large{\textbf{Electrical Engineering Graduate}}
\end{minipage}
\hfill
\begin{minipage}[t]{0.48\textwidth}
\begin{flushright}
\begin{normalsize}
e-mail Address: staltitia@yahoo.com\\
Mobile Number: 778-554-0903
\end{normalsize}
\end{flushright}
\end{minipage}
\rule{\textwidth}{1pt}

\vspace{10 mm}
\Large{\textbf{TECHNICAL SKILLS}} 
\normalsize
\vspace{2 mm}
\center
\begin{tabular*}{\textwidth}{@{\extracolsep{\fill} } l  l  l }
		\large\textbf{Electrical} & \large\textbf{Software} & \large\textbf{Coding}\\
		\hline
		& & \\
		Multimeter & MATLAB & C \\
		Breadboard & Quartus II & C++ \\
		Oscilloscope & Modelsim & Python \\
		Soldering & Microsoft Office & VHDL \\
		Digital/Logic circuit design & AutoCAD & ASM (NIOS II/8052) \\
		Circuit Analysis & KLayout & \\
\end{tabular*}
\flushleft

\vspace{5 mm}
\Large\textbf{CO-OP EXPERIENCE}
\normalsize

\vspace{5 mm}
\textbf{Honda Research Institute - Japan \hfill January 2013 - August 2013}	\\					Psychophysics Experiment Software Developer
\begin{itemize}
\itemsep0em
\item
	Developed experiment programs that provides audio-visual stimulus, records subject responses and saves results into a .csv
\item
	Built program to calculate subject's head position using OpenCV and spherical markers
\item
	Wrote MATLAB script to perform statistical analysis and plot graphs of multiple subjects for comparison
\item
	Tested hypotheses by performing experiments on multiple subjects in controlled environments
\item
	Presented findings before research staff upon completion of the work term
\end{itemize}
	
\vspace{5 mm}
\textbf{Epic 3D Tech \hfill May 2012 - August 2012} \\
Embedded Systems Engineer
\begin{itemize}
\itemsep0em
	\item
	Developed an Arduino program to control four stepper motors that move a two camera rig for automated 3D filming
	\item
	Wrote algorithm to convert physical position and angles to stepper motor steps based upon the physical dimensions of the camera rig
	\item
	Designed calibration and fail safe system using micro-switches and H-bridge circuit, and PCB to power the stepper motor
	\item
	Built control interface Android application that communicates with the Arduino micro-controller via a bluetooth module
	\item
	Documented and commented code written to allow for future development and improvement of the programs
\end{itemize}

\vspace{5 mm}
\textbf{UBC, Department of Physics and Astronomy \hfill September 2011 - December 2011} \\
PHYS 102 MRI Software Developer
\begin{itemize}
\itemsep0em
	\item
	Developed a server in C++ for an Arduino micro-controller to accept instructions according to a set protocol through a serial port
	\item
	Programmed a client in C to convert instructions in text format into the set protocol and send it over the serial port to the Arduino server, and to receive the data collected
	\item
	Built a graphic user interface (GUI) in Python to streamline the use of the C program, graph the data collected, and reconstruct the image of the sample being scanned by the MRI machine
	\item
	Wrote roughly 5000 lines of code across 3 programming languages throughout the course of the work term 
	\item
	Documented the code and wrote the instruction manual for the use of the program
	\item
	Designed templates for the gradient coils used to induce a magnetic gradient field for the imaging
\end{itemize}
	
\vspace{10 mm}
\Large{\textbf{TECHNICAL PROJECTS}}
\normalsize

\vspace{5 mm}
\textbf{Photonics System Modelling \hfill January 2014 - April 2014}

\begin{itemize}
\itemsep0em
	\item
	Designed silicon-based Mach-Zehnder Interferometers (MZI) and Fabry-Perot (FP) Interferometers using KLayout, a layout editor for .gds files
	\item
	Constructed numerical model in MATLAB for the ideal behaviour of fabricated photonics device
	\item
	Measured frequency response of devices using apparatus measuring output intensity during an input frequency sweep
	\item
	Adjusted model to match measured response of each respective device by varying device dimensions variables of MATLAB model
	\item
	Inferred physical device dimensions based upon adjustments to model
\end{itemize}	

\vspace{5 mm}
\textbf{SEM Control System \hfill September 2013 - April 2014}

\begin{itemize}
\itemsep0em
	\item
	Reverse engineered working scanning electron microscope (SEM) using documentation and schematics
	\item
	Designed system to interface analog circuitry to digital I/O using ADCs and DACs
	\item
	Developed control system using VHDL on an Altera DE II to read input from ADCs, provide output to DACs and communicate with a PC attached via serial
	\item
	Built GUI with a Python backend to allow user to control the SEM's current state, and to reconstruct images generated by the SEM
\end{itemize}

\vspace{5 mm}
\textbf{VHDL Implementation of Basic Computer \hfill January 2012 - April 2012}	

\begin{itemize}
\itemsep0em
\item	
Designed a basic computer that can compute the result of arithmetic operations upon values stored within registers, and store the result in a register
\item	
Programmed a Cyclone II FPGA using VHDL to make a basic computer consisting of a register bank, data path, arithmetic logic unit (ALU), and a finite state machine (FSM)
\item	
Planned a system where a series of switches control the operations of the computer, including type of operation, origin and destination registers, and input values, akin to an ASM instruction
\item	
Built a finite-state machine to control the execution of more complex operations, which require multiple steps to complete

\end{itemize}

\vspace{5 mm}
\textbf{Rover \hfill April 2011}							
\begin{itemize}
\itemsep0em
	\item
	Built a rover with a metal chassis, two DC motors, an 8052 microcontroller and three inductors capable of following a wire by detecting the AC current flowing through it
	\item
	Wired the rover's 8052 microcontroller to interpret the inductor's induced voltage with distance and orientation between the rover and the path indicated by the wire
	\item
	Programmed the microcontroller in C using a small device C compiler to detect and count wires perpendicular to the main track, interpret them as instructions which were then executed
	\item
	Worked in a group on the hardware, planning the algorithm and the writing of the code
	\item
	Tested and troubleshot the code written by running the rover on the track, observing behaviour and determining modifications required
\end{itemize}
	
\vspace{10 mm}
\Large{\textbf{EDUCATION}} 
\normalsize
\textbf{
\flushleft
University of British Columbia
\hfill
	September 2010 - May 2014
\\
	Bachelor of Applied Science - Electrical Engineering \\ 
	\hspace{10 mm}
	Nanotechnology \& Microsystems with Cooperative Education
}
%\section{ACTIVITIES AND INTERESTS}

\end{document}